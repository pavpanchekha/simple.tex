\documentclass{simple}

\title{Writing, Briefly\footnote{In the process of answering an email, I accidentally wrote a tiny essay about writing. I usually spend weeks on an essay. This one took 67 minutes—23 of writing, and 44 of rewriting.}}
\author{Paul Graham}
\date{March 2005}

\begin{document}
\maketitle

I think it's far more important to write well than most people realize. Writing
doesn't just communicate ideas; it generates them. If you're bad at writing and
don't like to do it, you'll miss out on most of the ideas writing would have
generated.

As for how to write well, here's the short version: Write a bad version 1 as
fast as you can; rewrite it over and over; cut out everything unnecessary;
write in a conversational tone; develop a nose for bad writing, so you can see
and fix it in yours; imitate writers you like; if you can't get started, tell
someone what you plan to write about, then write down what you said; expect
80\% of the ideas in an essay to happen after you start writing it, and 50\% of
those you start with to be wrong; be confident enough to cut; have friends you
trust read your stuff and tell you which bits are confusing or drag; don't
(always) make detailed outlines; mull ideas over for a few days before writing;
carry a small notebook or scrap paper with you; start writing when you think of
the first sentence; if a deadline forces you to start before that, just say the
most important sentence first; write about stuff you like; don't try to sound
impressive; don't hesitate to change the topic on the fly; use footnotes to
contain digressions; use anaphora to knit sentences together; read your essays
out loud to see (a) where you stumble over awkward phrases and (b) which bits
are boring (the paragraphs you dread reading); try to tell the reader something
new and useful; work in fairly big quanta of time; when you restart, begin by
rereading what you have so far; when you finish, leave yourself something easy
to start with; accumulate notes for topics you plan to cover at the bottom of
the file; don't feel obliged to cover any of them; write for a reader who won't
read the essay as carefully as you do, just as pop songs are designed to sound
ok on crappy car radios; if you say anything mistaken, fix it immediately; ask
friends which sentence you'll regret most; go back and tone down harsh remarks;
publish stuff online, because an audience makes you write more, and thus
generate more ideas; print out drafts instead of just looking at them on the
screen; use simple, germanic words; learn to distinguish surprises from
digressions; learn to recognize the approach of an ending, and when one
appears, grab it.

\end{document}
