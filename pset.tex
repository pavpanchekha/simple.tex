\documentclass{pset}
\title{Test Problem Set}
\author{Pavel Panchekha}
\class{18.702}
\begin{document}

\section{Problem 19.2}
\label{sec-1}


\part{1} Consider the set $\{ m r | r \in R \} = m R$.  It is the image
of the submodule $R$ under the map $\mu_m$, so it too is a submodule of
$M$.  It also contains $m$, so it is not $0$; thus it must be $M$.

\part{2} All fields are simple modules.  So if $M = R / \gm$, $M$ is
simple and clearly $\gm$ annihilates $M$.  On the other hand, if $M$
is simple and nonzero, choose nonzero $m \in M$ and consider $m R$.
$m R$ is a submodule of $M$ and is nonzero, so $m R = M$.  Then choose
a maximal ideal $\ga$ of $R$; $m \ga$ is a submodules of $M$, so since
$\ga \ne R$, $m \ga \ne M$ and thus $m \ga = 0$.  Thus $\ga$
annihilates $m$; since $m$ was arbitrary, $\ga = \Ann(M)$.
Furthermore, since $\ga$ annihilates $M$, $\gm$ maps to zero in $M$
and so $M \sub R / \ga$.  Since $\ga$ is maximal and $M$ nonzero, $M =
R / \ga$.

\part{3} Suppose $M$ has finite length but $M$ is not finitely
generated; then for any set of generators, there exists an element not
in the submodule generated by them.  Now, if $M$ has a finite length,
say due to chain $M_0 \sub M_1 \sub M_2 \sub \dots \sub M_l$.  By part (2), This implies
that $M_i / M_{i+1} = R / \gm_i$.
\section{Problem 19.4}
\label{sec-2}


If $M$ has finite length, the Jordan-H\"older theorem holds.  Thus the
$\gm \in \Supp(M)$ are maximal, so (1) implies (2).

The support of a module is a superset of the associated primes by
(17.16); so (2) implies (3).

Suppose the associated primes of $M$ are all maximal; apply (17.19),
yielding a chain $M \sup M_1 \sup \dots \sup M_l = 0$ where all quotients $M_i / M_{i-1}
= R / \gp_i$.  But since all $\gp_i$ are maximal, each of these quotients
is simple.  So this is a composition series, and thus its length,
which is finite, contributes the infimum that defines $\ell(M)$.  Thus
$\ell(M)$ is finite, and so (3) implies (1).

Finally, we must show that all of these imply that $\Ass(M) =
\Supp(M)$ are finite.  Both are clearly finite, as the Jordan-H\"older
theorem bounds the size of $\Supp(M)$ by $\ell(M)$ and (17.16) bounds
the size of $\Ass(M)$ by the size of $\Supp(M)$.  So we must simply
show that both are equal.  Consider an $\gm \in \Supp(M)$.  There exists
at least one $i$ such that $\gm = \Ann(M_{i-1} / M_i)$.  Choose a
concrete $m + M_i \in M_{i-1} / M_i$ and consider $m \in M_{i-1}$.  Then $\gm m =
0$, so $\gm \in \Ass(M)$.  So $\Ass(M) \supe \Supp(M)$ and by (17.16) $\Ass(M)
\sube \Supp(M)$, so we are done.
\section{Problem 19.9}
\label{sec-3}


Choose an element $m \in M$ and consider the map $\mu_m : k \to k m$.  This
map either maps all elements to zero or is bijective.  If it is
bijective, the $\ell(M) = \ell(k) + \ell(M / k m) = \ell(M / k m) +
1$.  Further, $\dim_k(M) = 1 + \dim_k(M / k m)$ for obvious reasons.  So
we may induct.  We need only deal with the case where the map $k \to k
m$ maps all elements to zero.  But since $1 \in k$, this case
corresponds to $m = 0$ for all choices $m$; thus $M = 0$, in which
case $\ell(M) = 0 = \dim_k(M)$.
  
\section{Problem 19.15}
\label{sec-4}


By (19.4), $\Ass(M)$ is finite and consists of maximal ideals.  Thus
$\Pi_{\gm \in \Ass(M)} \gm$ is defined and finite, and it clearly annihilates
all of $M$, so (1) implies (2).

Finally, we show (2) implies (4).  By (2), $\Ann(M) \sup \Pi \gm_i$.  We will
show that $R / \Pi \gm_i$ is Artinian; then $R / \Ann(M)$ must be Artinian
as well.  Let $M$ be annihilated by $\Pi_{i=1}^n \gm_i$ and let $N_k =
\Pi_{i=1}^{n-k} \gm_i$.  Then $R / N_0 \sup R / N_1 \sup \dots \sup R / N_{n-1} \sup 0$ is a
composition series, with each quotient being $m_{n+1-i}$.  Thus $R / N_0 =
R / \Pi \gm_i$ has finite length and is Artinian by (19.12).

(4) implies (3).  By (16.24), if $R / \Ann(M)$ is Artinian, every
prime ideal of $R / \Ann(M)$ is maximal, and thus any prime ideal
containing $\Ann(M)$ in $R$ is maximal.

By (13.5.3), $\gp$ contains the annihilator of $M$ if and only if $\gp \in
\Supp(M)$.  So if every prime containing the annihilator of $M$ is
maximal, all primes in $\Supp(M)$ are maximal and then by (19.4) $M$
has finite length.  So (3) implies (1).
\section{Problem 20.5}
\label{sec-5}


Let $M = \lang X, Y^2 \rang$, $N = \lang X^2, Y^2 \rang$, $K = k[X, Y]$.  $K$ has, by
the example (20.4), Hilbert Series $(1, 2, 3, 4, 5, \dots)$.  $M_n$ is
missing the terms $k$ and $k Y$, so has Hilbert Series $(0, 1, 3, 4,
5, \dots)$ instead of $(1, 2, 3, 4, 5, \dots)$; and $N_n$ is also missing terms
$k X$ and $k X Y$, so has Hilbert Series $(0, 0, 2, 4, 5, \dots)$.  So
clearly the two have different Hilbert Series.  On the other hand, the
Hilbert Function for $M$ is, for $n > 0$, $n+1$, and for $N$ is $n+1$
as well, for $n > 2$.  So for $n > 2$, $M$ and $N$ have the same
Hilbert Function and thus have the same Hilbert Polynomial.
\section{Problem 20.9}
\label{sec-6}


$\ell(R_n) = \ell(P_n) - \ell(\lang f \rang_n)$ by the additivity of length, and
since $\ell(P_n) = \binom{n + 2}{2}$ by (20.4) we need only find
$\ell(\lang f \rang_n)$.  If $f$ is of degree $d$, $f P_{n-d}$ consists of all
polynomials of degree $n$ divisible by $f$ and thus $\lang f \rang_n = f
P_{n-d}$ is a consistent grading for $\lang f \rang$.  Since $\ell(P_n) =
\binom{n+2}{2}$, we have $\ell(R_n) = \binom{n+2}{2} - \binom{n - d +
2}{2}$ (where the second term is zero for $n < d$).  This is $\frac12(2 n
d + 3d - d^2)$.
\end{document}
